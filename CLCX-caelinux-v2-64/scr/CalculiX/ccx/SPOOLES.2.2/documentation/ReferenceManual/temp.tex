%
%  main TeX file
%
\documentstyle[leqno,11pt,twoside]{report}

\textwidth      6.5 in
\textheight     8.5 in
\oddsidemargin  0.0 in
\evensidemargin 0.0 in
\topmargin      0.0 in
\baselineskip   20 pt
\parskip        3 pt plus 1 pt minus 1 pt

% \makeindex
\input psfig

\newcommand{\bfi}{{\bf i}}
\newcommand{\bfj}{{\bf j}}
\newcommand{\bnd}[1]{{\partial{#1}}}

\pagestyle{myheadings}
% \markboth
% {\quad \hrulefill \quad {\bf SPOOLES} : {\it DRAFT} \today \quad \hrulefill}
% {\quad \hrulefill \quad {\bf SPOOLES} : {\it DRAFT} \quad \today \hrulefill}
\markboth
{\quad \hrulefill \quad {\bf SPOOLES 2.0} : \today \quad \hrulefill}
{\quad \hrulefill \quad {\bf SPOOLES 2.0} : \quad \today \hrulefill}

\begin{document}

\bibliographystyle{plain}

\title{
       The Reference Manual for {\bf SPOOLES}, Release 2.0: \break
       An Object Oriented Software Library for Solving \break
       Sparse Linear Systems of Equations}
\author{
   Cleve Ashcraft\thanks{
   Boeing Shared Services Group,
   P. O. Box 24346,
   Mail Stop 7L-22,
   Seattle, Washington 98124,
   {\tt cleve.ashcraft@boeing.com}.
   This research was supported in part by the DARPA
   Contract DABT63-95-C-0122 and the DoD High Performance Computing
   Modernization Program Common HPC Software Support Initiative.}
 \and 
 David K. Wah\thanks{
   Boeing Shared Services Group,
   P. O. Box 24346,
   Mail Stop 7L-22,
   Seattle, Washington 98124,
   {\tt david.wah@pss.boeing.com}.
   This research was supported in part by the DARPA
   Contract DABT63-95-C-0122 and the DoD High Performance Computing
   Modernization Program Common HPC Software Support Initiative.}
% \and
%  Joseph W.H. Liu\thanks{
%     Department of Computer Science, York
%     University, North York, Ontario, Canada M3J 1P3.
%     This research was supported in part by the
%     Natural Sciences and Engineering Research Council of Canada
%     under grant A5509
%     and in part by the ARPA
%     Contract DABT63-95-C-0122.}
}
\date{\today}
\maketitle

\begin{abstract}
Solving sparse linear systems of equations is a common and important 
application of a multitude of scientific and engineering applications.
The {\bf SPOOLES} software package\footnote{
{\bf SPOOLES} is an acronym for {\bf SP}arse 
{\bf O}bject-{\bf O}riented {\bf L}inear {\bf E}quations {\bf S}olver.
}
provides this functionality with
a collection of software objects.
The first step to solving a sparse linear system is to find a good
low-fill ordering of the rows and columns.
The library contains several ways to perform this operation:
minimum degree, generalized nested dissection, and multisection.
The second step is to factor the matrix as a product of triangular
and diagonal matrices.
The library supports pivoting for numerical stability (when required),
approximation techniques to reduce the storage for and work to
compute the matrix factors, and the computations are based on BLAS3
numerical kernels to take advantage of high performance computing
architectures.
The third step is to solve the linear system using the computed
factors.
\par
The library is written in ANSI C using object oriented design.
Good design and efficient code sometimes conflict; 
generally we have preferred to cater to design. 
For large sparse matrices the serial code outperforms its FORTRAN 
predecessors, the reverse holds for moderate sized matrices 
or those that do not have good block structure.
The present release of the library contains a serial factorization 
and solve,
a multithreaded version using the Solaris and Posix thread packages,
and an MPI version.
There is considerable code overlap between the serial, threaded
and MPI versions.
\par
This release of the package is totally within the public domain; 
there are absolutely no licensing restrictions as with other
software packages.
The development of this software was funded by 
ARPA\footnote{ ARPA Contract DABT63-95-C-0122.} 
with the express purpose that
others (academic, government, industrial and commercial) could
easily incorporate the data structures and algorithms into
application codes.
All we ask is an acknowledgement in derivative codes and any
publications from research that uses this software.
And, we hope that any improvements will be communicated to others.
\end{abstract}
\par
% \input preface.tex

% \tableofcontents
% \listoffigures

\input partIntro.tex
% \input partUtil.tex
% \input partOrder.tex
% \input partNumeric.tex
% \input partMisc.tex
% \input partMPI.tex

% \bibliography{spooles}

% \input main.ind

\end{document}
